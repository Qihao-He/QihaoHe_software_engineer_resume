%-------------------------
% Resume in Latex
% Author : Qihao He
% License : MIT
%------------------------

\documentclass[letterpaper,11pt]{article}

\usepackage{latexsym}
\usepackage[empty]{fullpage}
\usepackage{titlesec}
\usepackage{marvosym}
\usepackage[usenames,dvipsnames]{color}
\usepackage{verbatim}
\usepackage{enumitem}
\usepackage[pdftex]{hyperref}
\usepackage{fancyhdr}
\usepackage{datetime}

\pagestyle{fancy}
\fancyhf{} % clear all header and footer fields
\fancyfoot{}
\renewcommand{\headrulewidth}{0pt}
\renewcommand{\footrulewidth}{0pt}

% Adjust margins
\addtolength{\oddsidemargin}{-0.375in}
\addtolength{\evensidemargin}{-0.375in}
\addtolength{\textwidth}{1in}
\addtolength{\topmargin}{-.5in}
\addtolength{\textheight}{1.0in}

\urlstyle{same}

\raggedbottom
\raggedright
\setlength{\tabcolsep}{0in}

% Sections formatting
\titleformat{\section}{
  \vspace{-4pt}\scshape\raggedright\large
}{}{0em}{}[\color{black}\titlerule \vspace{-5pt}]

%-------------------------
% Custom commands
\newcommand{\resumeItem}[2]{
  \item\small{
    \textbf{#1}{: #2 \vspace{-2pt}}
  }
}

\newcommand{\resumeSubheading}[4]{
  \vspace{-1pt}\item
    \begin{tabular*}{0.97\textwidth}{l@{\extracolsep{\fill}}r}
      \textbf{#1} & #2 \\
      \textit{\small#3} & \textit{\small #4} \\
    \end{tabular*}\vspace{-5pt}
}

\newcommand{\resumeSubItem}[2]{\resumeItem{#1}{#2}\vspace{-4pt}}

\renewcommand{\labelitemii}{$\circ$}

\newcommand{\resumeSubHeadingListStart}{\begin{itemize}[leftmargin=*]}
\newcommand{\resumeSubHeadingListEnd}{\end{itemize}}
\newcommand{\resumeItemListStart}{\begin{itemize}}
\newcommand{\resumeItemListEnd}{\end{itemize}\vspace{-5pt}}

%-------------------------------------------
%%%%%%  CV STARTS HERE  %%%%%%%%%%%%%%%%%%%%%%%%%%%%
\begin{document}

%----------HEADING-----------------
\begin{tabular*}{\textwidth}{l@{\extracolsep{\fill}}r}
    \textbf{{\Large Qihao He}} & Email: \href{mailto:qi.he@maine.edu}{qi.he@maine.edu}\\
    & Address: Orono, ME, 04469, USA\\
    & Work(landline): 2075812247, Mobile: 2075709742\\
    & Linkedin: https://www.linkedin.com/in/qihao-he-b37412b9/
  \end{tabular*}

%-----------EDUCATION-----------------
\section{Education}
  \resumeSubHeadingListStart
    \resumeSubheading
    {University of Maine}{Orono, ME, 04469, USA}
    {Master of Science in Computer Engineering (Thesis program); GPA: 3.5}{2014.09 -- 2018.05}

    \resumeSubheading
    {Shenzhen University}{Shenzhen, Guangdong, 518000, China}
    {Bachelor of Engineering in Electrical and Electronics;  GPA: 3.1}{2010.09 -- 2014.05}
  \resumeSubHeadingListEnd

%--------SKILLS------------
\section{Skills}
 \resumeSubHeadingListStart
    \resumeSubItem{Coding}
      {C (Proficient), Python, Java, MATLAB (Proficient)}
    \resumeSubItem{Skills}
      {Git, Embedded system, architecture, Shell/scripting, perf, Linux Performance counter, Altera Quartus, LaTeX}
    %   , power and timing measure, parallel programming, Visual Studio Code, Vim}
    \resumeSubItem{Languages}
      {Chinese (Native), English (Full professional), Cantonese (Native), Japanese}
 \resumeSubHeadingListEnd
 
 %-----------Honors/Awards-----------------
\section{Honors \& Awards}
\resumeSubHeadingListStart
 	\resumeSubItem{IEEE publishing in-process paper: Comparing Power and Energy Usage for Scientific Calculation with and without GPU Acceleration on a Raspberry Pi Model B+ and 3B (2018.03), ICOMP'18}
    {Explored the power and energy usage of the GPGPU on the Raspberry Pi and compared with them to the same calculation performed only on their CPUs. Developed 6 benchmarks to do 1D \& 2D FFT with and without GPU rendering using 3 mathematical libraries. The calculation was using single precision floating point numbers.}
    % https://csce.ucmss.com/cr/books/2018/AuthorsReport?ConferenceKey=ICM}
      
    \resumeSubItem{IEEE published paper: Raspberry Pi 2 B+ GPU Power, Performance, and Energy Implications (2017.03), 1 citation count, CSCI'16, Henry \& Grace Butler Award}
    {Correlated graphics processing, CPU load, power consumption and total energy. Measured the power consumption difference between GPU and CPU rendering by using different benchmarks both with and without the GPU rendering and an Adafruit USB Power Gauge.}
    % https://ieeexplore.ieee.org/document/7881331/}
      
    % \resumeSubItem{Henry \& Grace Butler Award (2016.08)}
    % {Award for the contribution in academic and research fields.}
\resumeSubHeadingListEnd

%-----------PROJECTS-----------------
\section{Projects}
\resumeSubHeadingListStart
    % ECE574: Cluster Computing
    \resumeSubItem{Parallelize computing using Message Passing Interface (MPI), CUDA, OpenMP, Pthreads for large-scale image segmentation on a Distributed-memory cluster system (2017.05)}
    {Developed a benchmark that used the specific matrices to multiply each pixel of the input image and change the pixel value for image segmentation. Used MPI, CUDA, OpenMP, Pthreads for large-scale image processing.}
    
    % COS515 Topics in Scientific Computing: Computer Simulation and Modeling, from Development to Display Physical models
    \resumeSubItem{Computer simulation and physical modeling, from development to display in Java (2016.12)}
    {Applying equations to physical modeling simulation. Implemented Euler, Euler-Cromer, Euler-Richardson, Runga-Kutta methods to analysis strength and weakness of different methods.Using wxMaxima to correlate dependent equations with coding.}
    
    % ECE523 Mathematical Methods in Electrical Engineering Duane Hanselman
    % \resumeSubItem{Application of mathematical and numerical methods to Electrical Engineering problems practice in MATLAB (2016.12)}
    % {Sufficient background in common numerical methods, interpolation problems, numerical integration problems, differentiation problems, ordinary differential equation problems, utilize algorithms appropriately and recognize their strength and weaknesses. Solving simple linear, nonlinear and dynamic electric circuits.}
    
    % ECE577 Fuzzy logic
    \resumeSubItem{Fuzzy logic controller for a rice cooker using MATLAB (2015.05)}
    {The controller used MATLAB Fuzzy logic toolkit to process the cooking while maintaining the temperature in a range instead of setting a threshold. Optimized Fuzzy logic for auto-adjustable process. Features: Fuzzy logic preheat, cooking process temperature controller.}
    
    % ECE473 Computer Architecture and Organization
    \resumeSubItem{A simple 5-stage pipeline processor in Altera Quartus II (2014.12)}
    {Built the virtual processor from scratch and downloaded to test board. Five stages: Intruction Fetch, Instruction Decode, Execution, Data Memory Access, Write Back Stage. Units included: Forwarding Unit, Hazard Detection Unit, Controls, ALU-Control Unit.}
    % Features: mathematical calculation, Read-Write Memory, overcome Data Hazards.}
    
    % ECE574: Cluster Computing CUDA
    % ECE523 Mathematical Methods in Electrical Engineering Duane Hanselman
    % COS515 Topics in Scientific Computing: Computer Simulation and Modeling, from Development to Display Physical models
    % ECE571 Advanced Microprocessor-Based Design Vincent Weaver
    % ECE573 Microprogramming and computer architecture
    % ECE473 Computer Architecture and Organization
    % ECE598: Advanced Operating Systems
    % ECE486 DSP
    % ECE515 Random Variables/Stoch Proc
    % ECE577 Fuzzy logic
    % ECE478 Industrial computer control (Minecraft relay logic)

  % \resumeSubItem{QuantSoftware Toolkit}
  %   {Open source python library for financial data analysis and machine learning for finance.}
\resumeSubHeadingListEnd
 
%-----------EXPERIENCE-----------------
\section{Experience}
  \resumeSubHeadingListStart
    \resumeSubheading
        {University of Maine}{Orono, ME, USA}
        {Volunteer Internship in the Advanced Computing Group of UMaine}{2016.01 -- 2017.12, 2018.09 -- Present}
        \resumeItemListStart
        \resumeItem{Maine Learn To Mod (NSF)}
        {Google Chrome browser Front-end User Interface virtual input keys and Back-end server script development; OpenStack (IAAS) dashboard project management; Correlate Apache Guacamole, Minecraft with OpenStack Virtual Machine.}
        \resumeItem{Visual wall}
        {Built the high-resolution 6K TV wall for high-density scientific data demonstration, including hardware assembling, system troubleshooting, and VNC software key-pairing.}
        \resumeItemListEnd

    \begin{tabular*}{0.97\textwidth}{l@{\extracolsep{\fill}}r}
    \textit{\small Teaching Assistant in Electrical \& Computer Engineering} & \textit{\small 2016.01 -- 2017.12}
    \end{tabular*}\vspace{-5pt}
        \resumeItemListStart
        % \resumeItem{Introduction to Programming for Engineers}
        % {Taught engineers to code, debug, compile, establish connections with test boards, run scripts, build circuit schematic, IC Pinouts, and graded them.}
        \resumeItem{Introduction to Robotics}
        {Graded robotics matrices mathematical calculations.}
        \resumeItemListEnd  
        
    	\resumeSubheading
        {Health Equity Alliance}{Bangor, ME, USA}
        {Volunteer Internship}{2018.05 -- 2018.08}
        \resumeItemListStart
        \resumeItem{Client Information Database}
        {Rebuilt the client information database using MS Office Access; essential knowledge of relational database, dependency between tables and database optimization rules for redundant data.}
        \resumeItemListEnd

        % \resumeSubheading
        % {China Great Wall Computer Shenzhen Company Limited}{Shenzhen, Guangdong, China}
        % {Assembly line Internship}{2013.06 -- 2013.09}
        % \resumeItemListStart
        % \resumeItem{Assemble desktop computer power supply}
        % {Proficient knowledge of the product's internal structure, quality control, assembly sequence, static protection, auto quick solder, storage optimization, fire safty, and product durability test.}
        % \resumeItemListEnd

    % \resumeSubheading
    %   {Google}{Mountain View, CA}
    %   {Software Engineer}{Oct 2016 - Present}
    %   \resumeItemListStart
    %     \resumeItem{Tensorflow}
    %       {TensorFlow is an open source software library for numerical computation using data flow graphs; primarily used for training deep learning models.}
    %     \resumeItem{Apache Beam}
    %       {Apache Beam is a unified model for defining both batch and streaming data-parallel processing pipelines, as well as a set of language-specific SDKs for constructing pipelines and runners.}
    %   \resumeItemListEnd
    
  \resumeSubHeadingListEnd

%-------------------------------------------
\end{document}
